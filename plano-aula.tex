%%%%%%%%%%%%%%%%%%%%%%%%%%%%%%%%%%%%%%%%%%%%%%%%%%%%%%%%%%%%%%%%%%%%%%%%
%%% Plano de aula do concurso do CEFET-SC
%%%%%%%%%%%%%%%%%%%%%%%%%%%%%%%%%%%%%%%%%%%%%%%%%%%%%%%%%%%%%%%%%%%%%%%%

\documentclass[11pt]{article}

\usepackage[utf8]{inputenc}
%\usepackage[latin1]{inputenc}
\usepackage[english,brazil]{babel}

\usepackage{indentfirst,enumerate,url,graphicx}
%\usepackage{pslatex,mathpazo}
\usepackage{type1cm}
\usepackage[a4paper,hmargin=2cm,vmargin=2cm]{geometry}
\newcommand{\eng}[1]{\foreignlanguage{english}{\textit{#1}}}
%\usepackage{helvet}
\pagestyle{empty}

\begin{document}

\begin{center}
  \LARGE\sf\bfseries Plano de Aula
\end{center}
\thispagestyle{empty}

\section{Identificação}

\begin{itemize}
\item Instituição: \parbox[t]{0.8\linewidth}{%
    Universidade do Estado de Santa Catarina}

\item Área: Lógicas e Semântica de Programas 

\item Professor: Rafael Castro Gonçalves Silva

\item Data: 10/07/2019

\end{itemize}

\section{Assunto}
\label{sec:assunto}

Sistemas de Tipos

\section{Objetivos}
\label{sec:objetivos}

Explicar o que são e para que servem sistemas de tipos em linguagens de programação.

\section{Conteúdo}
\label{sec:conteudo}

\begin{itemize}
  \item Sistemas de tipos em linguagens de programação.
  \item Breve visão histórica sobre Teoria de Tipos.
  \item Sistemas de tipos formais em linguagens de programação.
\end{itemize}

\section{Bibliografia}
\label{sec:bibliografia}

\begin{enumerate}
\item PIERCE, B. C. Types and Programming Languages. 1st. ed. [S.l.]: The MIT Press, 2002. 
\item SEBESTA, R. W. Conceitos de Linguagens de Programaç ao. Porto Alegre, RS: Editora Porto Alegre, 2003.
\end{enumerate}
\end{document} 
