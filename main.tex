\documentclass{beamer}

\usepackage{graphicx,hyperref,udesc,url}
\usepackage[latin1]{inputenc}
\usepackage{bussproofs}
%\usepackage[T1]{fontenc}
\usepackage{booktabs}
\usepackage[portuges]{babel}


\title[Extraction of Programs from Proofs]{Extraction of Programs from Proofs}

\author[Rafael Castro]{
    Rafael Castro\\\medskip
    {\small \url{rafaelcgs10@gmail.com}}}

    \institute[UDESC]{
        Departamento de Ci\^encia da Computa\c{c}\~ao \\
            Centro de Ci\^encias e Tecnol\'ogicas\\
            Universidade do Estado de Santa Catarina}

\date{07/03/2018}

\begin{document}

\begin{frame}
\titlepage

\end{frame}

\section{Isomorfismo de Curry-Howard}

\begin{frame}
\frametitle{Sistemas de Provas}
\begin{itemize}
    \item Sistemas/C�lculos de provas servem para construir provas de uma maneira muito formal.
    \item S�o uma cole��o de regras que explicam como derivar novas f�rmulas.
    \item Um sistema de prova pode ser utilizado na formaliza��o de diversas l�gicas,
      como L�gica Proposicional e L�gica de Predicados.
    \item Os principais sistemas de provas s�o a Dedu��o Natural e o C�lculo de Hilbert.
\end{itemize}
\end{frame}

\begin{frame}[fragile]
\frametitle{Dedu��o Natural}
\begin{center}
        \begin{prooftree}
            \AxiomC{}
            \RightLabel{$(ax)$}
            \UnaryInfC{$u : A$}
        \end{prooftree}
\end{center}

\begin{center}
\begin{minipage}{.4\textwidth}
        \begin{prooftree}
            \AxiomC{$A \rightarrow B$}
            \AxiomC{$A$}
            \RightLabel{$(\rightarrow^-)$}
            \BinaryInfC{$B$}
        \end{prooftree}
\end{minipage}
\begin{minipage}{.4\textwidth}
        \begin{prooftree}
            \AxiomC{$B$}
            \RightLabel{$(\rightarrow^+) \: u : A$}
            \UnaryInfC{$A \rightarrow B$}
        \end{prooftree}
\end{minipage}
\end{center}
        
\begin{center}
\begin{minipage}{.3\textwidth}
        \begin{prooftree}
            \AxiomC{$A \wedge B$}
            \RightLabel{$(\wedge_l^-)$}
            \UnaryInfC{$A$}
       \end{prooftree}
\end{minipage}
\begin{minipage}{.3\textwidth}
        \begin{prooftree}
            \AxiomC{$  A \wedge B$}
            \RightLabel{$(\wedge_r^-)$}
            \UnaryInfC{$B$}
       \end{prooftree}
\end{minipage}
\begin{minipage}{.3\textwidth}
        \begin{prooftree}
            \AxiomC{$A$}
            \AxiomC{$B$}
            \RightLabel{$(\wedge^+)$}
            \BinaryInfC{$A \wedge B$}
       \end{prooftree}
\end{minipage}
\end{center}

\begin{center}
\begin{minipage}{.5\textwidth}
        \begin{prooftree}
            \AxiomC{$A \vee B$}
            \AxiomC{$A \rightarrow C$}
            \AxiomC{$B \rightarrow C$}
            \RightLabel{$(\vee^-)$}
            \TrinaryInfC{$C$}
       \end{prooftree}
\end{minipage}
\end{center}

\begin{center}
\begin{minipage}{.3\textwidth}
        \begin{prooftree}
            \AxiomC{$A$}
            \RightLabel{$(\vee_l^+)$}
            \UnaryInfC{$A \vee B$}
       \end{prooftree}
\end{minipage}
\begin{minipage}{.2\textwidth}
        \begin{prooftree}
            \AxiomC{$B$}
            \RightLabel{$(\wedge_r^+)$}
            \UnaryInfC{$A \vee B$}
       \end{prooftree}
\end{minipage}
\end{center}

\begin{center}
\begin{minipage}{.3\textwidth}
        \begin{prooftree}
            \AxiomC{$\bot$}
            \RightLabel{$(efq)$}
            \UnaryInfC{$A$}
        \end{prooftree}
\end{minipage}
\begin{minipage}{.3\textwidth}
        \begin{prooftree}
            \AxiomC{$\neg \neg A$}
            \RightLabel{$(raa)$}
            \UnaryInfC{$A$}
        \end{prooftree}
\end{minipage}
\end{center}
\end{frame}

\begin{frame}
\frametitle{Exemplo de Prova em DN}
\end{frame}

\begin{frame}
\frametitle{C�lculo Lambda}
\end{frame}

\begin{frame}
\frametitle{Exemplo de Computa��o em CL}
\end{frame}

\begin{frame}
\frametitle{C�lculo Lambda Tipado}
\end{frame}

\section{Extra��o de Programas}

\begin{frame}
\frametitle{Realizability}
\end{frame}

\section{Extra��o em Minlog}

\begin{frame}
\frametitle{Realizability}
\end{frame}


\end{document}