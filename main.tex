\documentclass{beamer}

\usepackage{graphicx,hyperref,udesc,url}
\usepackage[latin1]{inputenc}
%\usepackage[T1]{fontenc}
\usepackage{booktabs}
\usepackage[portuges]{babel}


\title[Extraction of Programs from Proofs]{Extraction of Programs from Proofs}

\author[Rafael Castro]{
    Rafael Castro\\\medskip
    {\small \url{rafaelcgs10@gmail.com}}}

    \institute[UDESC]{
        Departamento de Ci\^encia da Computa\c{c}\~ao \\
            Centro de Ci\^encias e Tecnol\'ogicas\\
            Universidade do Estado de Santa Catarina}

\date{07/03/2018}

\begin{document}

\begin{frame}
\titlepage

\end{frame}

\section{Isomorfismo de Curry-Howard}

\begin{frame}
\frametitle{Sistemas de Provas}
\begin{itemize}
    \item Sistemas/C�lculos de provas servem para construir provas de uma maneira muito formal.
    \item S�o uma cole��o de regras que explicam como derivar novas f�rmulas.
    \item Um sistema de prova pode ser utilizado na formaliza��o de diversas l�gicas,
      como L�gica Proposicional e L�gica de Predicados.
    \item Existem diversos sistemas de provas: Dedu��o Natural e C�lculo de Hilbert.
\end{itemize}
\end{frame}

\begin{frame}
\frametitle{Dedu��o Natural}
\end{frame}

\begin{frame}
\frametitle{Exemplo de Prova em DN}
\end{frame}

\begin{frame}
\frametitle{C�lculo Lambda}
\end{frame}

\begin{frame}
\frametitle{Exemplo de Computa��o em CL}
\end{frame}

\begin{frame}
\frametitle{C�lculo Lambda Tipado}
\end{frame}


\section{Extra��o de Programas}
\begin{frame}
\frametitle{Realizability}
\end{frame}
